\documentclass[a4paper, 12pt, titlepage, utf8]{extarticle}

\usepackage[utf8]{inputenc}
\usepackage[english,russian]{babel}

% TODO титульник и оформление
%	содержание
%	список литературы
%	спеллчек

\begin{document}

\tableofcontents
\pagebreak

% -------------------- Введение --------------------
\section{Введение}	% название данной секции стандартно

Исследования в области биоинформатики приобретают высокую актуальность в наши дни. Современные вычислительные мощности позволяют обрабатывать большие объемы данных характерные для биологических объектов, и выполнять сложные математические расчеты за приемлемое время. Биоинформатика позволяет исследователям эффективно автоматизировать обработку экспериментальных данных, а в ряде случаев решать поставленные задачи аналитически, не прибегая к дорогостоящим и длительным экспериментам.

Одним из важнейших напралений в молекулярной биологии и фармацевтике является структурная биология, объектами изучения которой являются структуры биологических макромолекул, в частности белков и нуклеиновых кислот. Свойства и назначение этих соединений во многом определяются их структурой. Поэтому исследование структуры биополимеров позволяет лучше понимать процессы, протекающие в живом организме, находить связи между биологическими объектами и выяснять причины возникновения возможных паталогий и заболеваний а значит создавать новые высокоэффективные препараты. Инструменты анализа созданные с помощью вычислительной биологии способствуют достижению этих целей и являются неотемлемой частью современной науки. Существует множество методов автоматизации анализа структур таких как трехмерная визуализация, построение структурного выравнивания, предсказание вторичной и третичной структур белка из кодирующей последовательности, моделирование функциональности и взаимодействия биополимеров и другие.

Необходимым средством анализа биологических структур является инструмент построения структурного выравнивания, позволяющий сравнивать структуры между собой. Целью данной работы является разработка такого инструмента на базе проекта с открытым исходным кодом Unipro UGENE. UGENE -- это комплексный инструмент автоматизации биологических исследований, поддерживающий большое количество средств анализа последовательностей ДНК, РНК и аминокислот. UGENE является мультиплатформенным приложением, позволяющим достичь максимальной производительности работы алгоритмов за счет платформо-зависимых оптимизаций и адаптирования алгоритмов для выполнения в многопоточных средах. В настоящий момент в UGENE уже реализованы средства визуализации трехмерных структур а так же некоторые методы структурного анализа, таким образом реализация инструмента построения структуроного выравнивания является закономерным шагом в развитии проекта в данной области.

Использование разработанного компонента в совокупности с уже существующим инструментарием UGENE даст возможность эффективней решать сложные многоэтапные задачи, связанные с макромолекулярными структурами, что несомненно повысит результативность работы исследователей в этой области.

\section{Описание предметной области}	% название данной секции стандартно

\subsection{О структуре белка}
Третичная структура это трехмерная модель молекулы белка (или все же биополимера?). Структура молекулы играет важнейшую роль. Средний протеин содержит 300? остатков а значит всего 20\^300(это кстати больше чем атомов во вселенной) вариантов из которых в живых организмах присутствуют очень маленькое число. Вариантов сворачивания белка тоже до черта. Структурная биология важное направление в биологии. Она активно развивается. Сейчас эти структуры получают такими: NMR, X-Ray методами а так же моделируют (предсказывают). Сегодня выпускают по N (книга Structural Bioinformatics) в месяц и уже известно M? структур.
Клевое слово метаинформация

\subsection{О формате PDB}
и о базе тоже
В него можно писать выравнивания -- PROFIT!

\subsection{О структурном выравнивании}
Структурное выравнивание -- это метод определения схожести структур по их третичной структуре.% зафиксировать уже это определение % 
Структурное выравнивание является важным инструментом имеет несколько важных применений:
	(взято из книги Structural Bioinformatics)
	- Классификация белков по группам, и создание библиотек паттернов для последующего аннотирования
	- Сравнение белка с известной функцией против неизвестного белка может помочь определить его функциональное назначение 
	- Методы предсказания структуры требуют сравнения предсказанных структур с заранее известными шаблонами
	- Структурное выравнивание может выявить связь между белками неявную из выравнивания последовательностей (там еще есть но пока непонятно)
	- ???
	- PROFIT!

Тут надо как-то аккуратно рассказать о том что не все йогурты одинаково полезны, например есть алгоритмы тупо сравнивающие облака точек одинакого размера, это то любой школьник может !привет моя реализация! и их-то как грязи; а есть те которые стравнивают вообще ни разу непохожие структуры вот они то и крутые, есть варианты как в химере [и ссылочку на статью] которые строят на основе выравнивания последовательностей и тоже там какие-то результаты получают

Структурное выравнивание это NP полная задача, все существующие методы эвристические, кроме того разные методы могуть давать разные результаты.
Суть метода (тут картинку из презентации, не зря же я её корячился рисовал)

\section{Обзор существующих решений}
Существуют как пакеты специально предназначенные для построения выравниваний (тут оборзеть\^Wобозреть несколько таких тулов +ссылка на страницу сравнения и википедию) так и присутствует в разноплановых визуализаторах в виде дополнительной функциональности (парочку тоже накинуть).
Первые испытывают проблемы с интеграцией с другими инструментами (это у кости расписано) 
Недостатки вторых надо у кости писать.

Существует множество алгоритмов для решения данной задачи. 

\section{Постановка задачи}	% название данной секции стандартно

Цель работы -- заиметь такой еба алгоритм в виде готового инструмента у себя в UGENE. 
Задачи.
Проанализировать готовые алгоритмы и выбрать подходящий с учетом этих -> (только C/C++, только открытый код, только хардкор) требований. Запилить абстрактный интерфейс для алгоритма структурного выравнивания. Засовать выбранынй алгоритм в плагин. Заделать гуевину чтобы даже полный идиот мог сделать выравнивание а не как во всех остальных тулах где только кодер который их разработал может это сделать. Сделать визуализацию построенного выравнивания на базе существующего BioStruct3DView. (вот вопрос: писать те задачи которые остались не сделанными или хитрожопо пропустить?) Сделать импорт и экспорт выравнивания в формат совместимый с другими инструментами. Покрыть все это дело тестами и вообще тщательно протестировать.


(Тут вставляестя простыня с тулами/библиотеками где в них нужно оыскать максимальное количество недостатков)

Потом надо подытожить недостатки и пообещать что наше решение ими страдать не будет.

%%\section{Анализ задачи} Пока непонятно, вроде это уже написано выше в постановке.

\section{Реализация}
\subsection{Внутреннее представление данных UGENE}
UGENE имеет внутреннее представление структуры молекулы не зависящее от конкретного формата а так же DocumentFormat что позволяет абстрагироваться от конкретного формата а так же сравнительно легко добавлять поддержку новых форматов.
Про BioStruct3D +диаграмма

\subsection{Абстрактный интерфейс}
Анализ показал что для алгоритмов выравниваня можно выделить абстрактный интерфейс StructuralAlignmentAlgorithm. Каждый алгоритм должен имплементировать его.
Абстрактный интерфейс +диаграмма

Так же в UGENE существует абстракция Task которая унифицирует и позволяет в некоторых случаях прозрачно распараллеливать UGENE.
Вычислительная задача +диаграмма

\subsection{Subset}
Часто представляет интерес выравнивание не молекул целиком а только их частей. Соответственно у пользователя должна быть возможность задавать интересующие его регионы. Для этого был добавлен тип данных BioStruct3DSubset. И соответствующий графический интерфейс для его удобного задания. Так же учитывается что одна и та же молекула может иметь несколько моделей (так называемые NMR ансамбли) соответственно можно указать интересующую модель или даже с
Написать про subset и зачем он нужен. 
Про editor subset'a
+screenshot

\subsection{PTools}
Таки был выбран PTools надо обосновать почему: С++, открытый код, небольшой размер, документация -- вот почему.

Про адаптацию PTools к разным компиляторам например.
Тесты которые фиксируют отсутствие регрессий

Про то что типы данных разные у нас и у PTools и что конвертер надо.
Так же есть тесты подтвердающие что результаты одинаковые для их и для нашего загрузчика

\subsection{Визуализация}
Тут про мучительный рефакторинг того что уже было BioStruct3DView \& BioStruct3D core.
+screenshot


\section{Заключение и выводы}

Честно говоря это только перывй шаг и алгоритм выбран детский, но тем не менее он работает и в отличие от исходного вида им может пользоватсья каждый кто пожелает. Так же есть визуализация что не может не радовать. Так же этот алгоритм годится для обкатки API и всей обвязки (визуализация, импорт/экспорт). Ну и код у нас открытый велкам запиливать свои алгоритмы. Так же есть сообщество пользователей с которыми мы нацелены работать и пожелания которых всегда приоритетны. 

Так же я получил бесценный опыт промышленного шпионажа\^Wкодирования со всеми вытекающими и намерен развивать то что сваял дальше, благо, есть куда.

\section{Список литературы}

Книга Structural Bionformatics
Книга мат методы биоинформатики (надо найти, скачать, прочитать, мож че вставить даже)
Статья про Химеру
Статья про хитрожопое выравнивание в химере
Статья про PTools
Статья про UGENE
Просмотреть список литературы у википедии

\end{document}
